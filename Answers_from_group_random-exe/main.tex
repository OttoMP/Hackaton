\documentclass{article}
\usepackage{amsmath}
\usepackage{amssymb}
\usepackage{titlesec}
\usepackage[a4paper, top=2.5cm, bottom=2.5cm, left=1.5cm, right=1.5cm]{geometry}
\begin{document}
\section*{Introduction to Applications of Quantum Computing to Quantum Chemistry}
Both the Exercises and Challenge are uploaded as Jupyter notebooks.
\section*{Bose-Einstien Condensates and the Involvement in Advances for New Technologies}

%\subsection{Interaction between atoms and low energy limit(Exercise)}

\textbf{(1) [Exercise] Interactions between atoms and the low-energy limit.}\\

\textbf{a)}\\
Scattering is taken into account by introducing a potential $U(\bold{r})$ that acts on an initial state, and we obtain a final state, in a finite range $(r<R)$ with hamiltonian
\[
H = H_0 + U(\mathbf{r})
\]
the result for the asymptotic behavior of the wave function is (for large r)
\[
\psi(\mathbf{r})\rightarrow \frac{1}{(2\pi)^{3/2}} \left[ e^{i\mathbf{k.r}}+\frac{e^{ik.r}}{r}f(\mathbf{k},\mathbf{k'})\right]
\]
where
\[
f(\mathbf{k},\mathbf{k'})=-\frac{m(2\pi)^3}{2\pi\hbar^2}\int d^3r'\langle \mathbf{k'}|\mathbf{r'}\rangle U(\mathbf{r'})\langle \mathbf{r'}|\mathbf{\psi}\rangle 
\]
To obtain the last wave function, we assumed the scattering potential $U(\mathbf{r})$ to be real, finite-ranged, local, and that it is spherically symmetric. Hence, we can write $U(\mathbf{r})$ $= U(r)$. In the scattering region $(0 < r < R)$, we can write the Schrödinger equation as
\[
 -\frac{\hbar^2}{2m} \nabla^2\psi+U(r)\psi = E\psi
\]
Due to the spherical symmetry of $U(r)$ the Laplacian can be expressed in spherical coordinates
\[
 \left(-\frac{\hbar^2}{2m}\frac{1}{r}\frac{\partial^2}{\partial r^2}r+\frac{L^2}{2mr^2} +U(r)\right)\psi(r,\theta,\phi) = E\psi(r,\theta,\phi)
\]
where $L$ is the angular momentum operator. From the last equation we can see that the particle is subject to the action of an effective potential
\[
U_{eff}(r)=U(r)+\frac{\hbar^2}{2m}\frac{l(l+1)}{r^2}
\]
where the second term on the right-hand side is a
repulsive centrifugal barrier. It is absent for $l = 0$.. 

For low-energy scattering, in terms of $k$ and $R$ is:
\[
k R \ll 1
\]
where the reduced wavelength $\frac{\lambda}{2\pi}=\frac{1}{k}$.


If the energy is close to zero, the particle cannot overcome the centrifugal barrier and results with $l>0$ are not important and the component with $l=0$ is dominant to understand the low-energy scattering. Hence it is most relevant.
\vspace{1cm}\\
\textbf{b)} \underline{S-wave scattering length}:
\vspace{0.5cm}\\
S-wave scattering length is the parameter that characterizes the effective range of interaction in case of the low energy limit of the incident wave, i.e., for $l = 0$ term in the partial wave analysis referred to as “s-wave”.
\vspace{1cm}\\
\textbf{c)} The scattering length is given by\\
\[
a=\frac{m_r}{2\pi\hbar^2}\int U(\mathbf{r})d^3r
\]
For $U=U_0\delta(\mathbf{r}-\mathbf{r}')$:\\
\[
a=\frac{m_r}{2\pi\hbar^2}\int U_0\delta(\mathbf{r}-\mathbf{r}')d^3r
\]
\[
a=\frac{m_r}{2\pi\hbar^2}U_0\int \delta(\mathbf{r}-\mathbf{r}')d^3r
\]
\[
a=\frac{m_r}{2\pi\hbar^2}U_0\int \delta(x-x')\delta(y-y')\delta(z-z')dxdydz
\]
\[
a=\frac{m_r}{2\pi\hbar^2}U_0
\]
\[
U_0=\frac{2\pi\hbar^2a}{m_r}
\]
therefore
\[
U=\frac{2\pi\hbar^2a}{m_r}\delta(\mathbf{r}-\mathbf{r}')
\]
%\subsection{The Gross-Pitaevskii equation(Exercise)}
\textbf{(2) [Exercise] The Gross-Pitaevskii equation.}\\

\textbf{a)} We have the Hamiltonian as


\[
H=\sum_{i=1}^N[\frac{\mathbf{p}_i^2}{2m}+V(\mathbf{r}_i)]+U_0\sum_{i<j}\delta(\mathbf{r}_i-\mathbf{r}_j)
\]
and the wave function as


\[
\Psi(\mathbf{r}_1,\mathbf{r}_2,...,\mathbf{r}_N)=\prod_{i=1}^N\phi(r_i)
\]
Expectation value,
\[
 E= \int \Psi^*(\mathbf{r}_1,\mathbf{r}_2,...,\mathbf{r}_N)H\Psi(\mathbf{r}_1,\mathbf{r}_2,...,\mathbf{r}_N)\prod_i d^3r_i
\]
\[
=\int \prod_i \left( \phi^*(r_i) \left( \sum_{i=1}^N[\frac{\mathbf{p}_i^2}{2m}+V(\mathbf{r}_i) \right) + U_0 \sum_{j<k} \delta(\mathbf{r}_i-\mathbf{r}_j) \right) \phi(r_i) d^3r_i
\]
\[
\text{with}~
\mathbf{p}_i = i\hbar \frac{\partial}{\partial x_i} + i\hbar \frac{\partial}{\partial y_i} + i\hbar \frac{\partial}{\partial z_i}
\]
\[
= \prod_i \int \phi^*(\mathbf{r}_i) \left( \sum_{j=1}^{N} \left( -\frac{\hbar^2}{2m} \nabla_j^2 + V(\mathbf{r}_j) \right) + U_0 \sum_{j<k} \delta(\mathbf{r}_i-\mathbf{r}_j) \right) \phi(\mathbf{r}_i) d^3r_i
\]
\[\text{As} \int d^3r_i |\phi(r_i)|^2 = 1 \quad \forall i\]\\
we get:
\[
= \sum_{j=1}^{N} \int \phi^*(\mathbf{r}_j) \left( -\frac{\hbar^2}{2m} \nabla_j^2 + V(\mathbf{r}_j) \right) \phi(\mathbf{r}_j) d^3r_j
\]
\[
+ \sum_{j=1}^N\sum_{j<k} \int \phi^*(\mathbf{r}_j) \phi^*(\mathbf{r}_k) U_0 \delta(\mathbf{r}_j - \mathbf{r}_k) \phi(\mathbf{r}_j) \phi(\mathbf{r}_k) d^3r_j d^3r_k
\]
\textbf{(b)} As each $\phi_i$ will give the same contribution to the integral, we have:
\[
E = N \int \phi^*(\mathbf{r}) \left( -\frac{\hbar^2}{2m} \nabla^2 + V(\mathbf{r}) \right) \phi(\mathbf{r}) d^3r
\]
\[
+ \frac{U_0 N(N-1)}{2} \int \phi^*(\mathbf{r}') \phi^*(\mathbf{r}) \delta(\mathbf{r} - \mathbf{r}') \phi(\mathbf{r}') \phi(\mathbf{r}) d^3r d^3r'
\]

\[
= N \int \left( -\frac{\hbar^2}{2m} \phi^*(\mathbf{r})\nabla^2 \phi(\mathbf{r}) + V(\mathbf{r}) |\phi(\mathbf{r})|^2 + \frac{U_0 N(N-1)}{2} |\phi(\mathbf{r})|^4 \right) d^3r
\]
We have 
\[
\Psi(\mathbf{r}) = \sqrt{N} \phi(\mathbf{r})
\]
so:
\[
= \int \left( -\frac{\hbar^2}{2m} \Psi^*(\mathbf{r})\nabla^2 \Psi(\mathbf{r}) + V(r) |\Psi(\mathbf{r})|^2 + U_0 \left(\frac{1}{2} - \frac{1}{2N} \right) |\Psi(\mathbf{r})|^4 \right) d^3r
\]
For large $N$, ignoring $\frac{1}{N}$ terms:
\[
E = \int \left( -\frac{\hbar^2}{2m} \Psi^*(\mathbf{r})\nabla^2 \Psi(\mathbf{r}) + V(r) |\Psi(\mathbf{r})|^2 + \frac{U_0}{2} |\Psi(\mathbf{r})|^4 \right) d^3r
\]
Using the method of Lagrange multipliers, we have to minimize the quantity $E - \mu N$, with $\mu$ being the chemical potential, and:
\[
N = \int |\Psi(\mathbf{r})|^2 d^3r
\]
Seeing its variation w.r.t. $\Psi^*(\mathbf{r})$, we get:
\begin{align*}
&\frac{\partial}{\partial \Psi^*(\mathbf{r})} \left( E - \mu N \right) =\\
&\quad \frac{\partial}{\partial \Psi^*(\mathbf{r})} \int \Big( -\frac{\hbar^2}{2m} \Psi^*(\mathbf{r})\nabla^2 \Psi(r) + V(\mathbf{r})\Psi^*(r)\Psi(\mathbf{r}) + \frac{U_0}{2}\Psi^*(\mathbf{r})\Psi(\mathbf{r})\Psi^*(\mathbf{r})\Psi(\mathbf{r})\\
& \qquad \qquad \qquad \qquad \qquad \qquad \qquad \qquad \qquad \qquad \qquad- \mu |\Psi(\mathbf{r})|^2 \Big) = 0
\end{align*}

\[
\Rightarrow \left( -\frac{\hbar^2}{2m} \nabla^2 \Psi(\mathbf{r}) + V(\mathbf{r}) \Psi(\mathbf{r}) + U_0 |\Psi(\mathbf{r})|^2 \Psi(\mathbf{r}) - \mu \Psi(\mathbf{r}) \right) d^3r = 0
\]

Which leads to:
\[
\left( -\frac{\hbar^2}{2m} \nabla^2 + V(\mathbf{r}) + U_0 |\Psi(\mathbf{r})|^2 \right) \Psi(\mathbf{r}) = \mu \Psi(\mathbf{r})
\]

This is the time-independent Gross-Pitaevskii equation.
\vspace{1cm}\\
\textbf{(c)} For uniform Bose gas, $\Psi(\mathbf{r})$ doesn't have a spatial dependence, so:
\[
\nabla^2 \Psi(\mathbf{r}) = 0
\]
Also, there is no trapping potential for a uniform Bose gas. Thus, $V(\mathbf{r}) = 0$. So the G.P.E. becomes:
\[
U_0 |\Psi(\mathbf{r})|^2 = \mu
\]
%\subsection{Computational Project(Challenge)}

\textbf{(3) [Challenge] Computational project.} \\

The Challenge has been uploaded as a Jupyter notebook


\section*{ Prospects and Challenges for Quantum Machine Learning}

\textbf{(a)}[\textbf{Exercise}]
We have \( Id, X,\cdot\)(dot product).

It's a group if :


\textbf{Closure}
\( \forall a, b \in V \), \( a \cdot b \in V \)

So:
\[
Id \cdot X = X \quad \text{and} \quad X \cdot Id = X \quad \text{where} \quad X \in V.
\]


\textbf{Associativity}
For all \( a, b, c \in V \):
\[
(a \cdot b) \cdot c = a \cdot (b \cdot c)
\]

Example checks for associativity:


\textbf{1.)}Given: \( a = Id \), \( b = X \), \( c = Id \)

\[
\begin{aligned}
(Id \cdot X) \cdot Id &= X \cdot Id = X \\
Id \cdot (X \cdot Id) &= Id \cdot X = X
\end{aligned}
\quad \} \text{ All are equal}
\]


\textbf{2.)}Given: \( a = Id \), \( b = X \), \( c = X \)

We know:
\[
\begin{pmatrix} 
0 & 1 \\
1 & 0 
\end{pmatrix}
\begin{pmatrix} 
0 & 1 \\
1 & 0 
\end{pmatrix}
= Id
\]
Then:
\[
\begin{aligned}
(Id \cdot X) \cdot X &= X \cdot X = Id \\
Id \cdot (X \cdot X) &= Id \cdot Id = Id
\end{aligned}
\quad \} \text{ Are equal}
\]

\textbf{3.)}
Given: \( a = Id \), \( b = Id \), \( c = Id \)

\[
\begin{aligned}
(Id \cdot Id) \cdot Id &= Id \cdot Id = Id \\
Id \cdot (Id \cdot Id) &= Id \cdot Id = Id
\end{aligned}
\quad \} \text{ Are equal}
\]

\textbf{4.)}
Given: \( a = Id \), \( b = Id \), \( c = X \)

\[
\begin{aligned}
(Id \cdot Id) \cdot X &= Id \cdot X = X \\
Id \cdot (Id \cdot X) &= Id \cdot X = X
\end{aligned}
\quad \} \text{ Are equal}
\]

\textbf{5.)}
Given: \( a = X \), \( b = X \), \( c = Id \)

\[
\begin{aligned}
(X \cdot X) \cdot Id &= Id \cdot Id = Id \\
X \cdot (X \cdot Id) &= X \cdot X = Id
\end{aligned}
\quad \} \text{ Are equal}
\]

\textbf{6.}
Given: \( a = X \), \( b = X \), \( c = X \)

\[
\begin{aligned}
(X \cdot X) \cdot X &= Id \cdot X = X \\
X \cdot (X \cdot X) &= X \cdot Id = X
\end{aligned}
\quad \} \text{ Are equal}
\]

\textbf{7.)}
Given: \( a = X \), \( b = Id \), \( c = Id \)

\[
\begin{aligned}
(X \cdot Id) \cdot Id &= X \cdot Id = X \\
X \cdot (Id \cdot Id) &= X \cdot Id = X
\end{aligned}
\quad \} \text{ Are equal}
\]

\textbf{8.)}
Given: \( a = Id \), \( b = Id \), \( c = X \)

\[
\begin{aligned}
(Id \cdot Id) \cdot X &= Id \cdot X = X \\
Id \cdot (Id \cdot X) &= Id \cdot X = X
\end{aligned}
\quad \} \text{ Are equal}
\]
Thus all elements are associative.


\textbf{Identity Element}
For all \( a \in V \), there exists \( e \in V \) called the identity of \( a \), such that:
\[
a \cdot e = e \cdot a = a
\]

\textbf{1.)} If a=Id
Then:
\[
Id \cdot Id = Id.
\]

\textbf{2.)} If a=X
Then 
\[
X \cdot Id =Id \cdot X=X
\]
Thus  \( e=Id \) is the identity element of the group.
\\
\textbf{Inverse Element}
For all \( a \in V \), there exists \( b \in V \) called the inverse of \( a \), such that:
\[
a \cdot b = b \cdot a = e
\]
where \( e=Id\) is the identity element.

\textbf{1.)} If a=Id
We assume that \( b = Id \), thus:
\[
Id \cdot Id = Id.
\]

\textbf{2.)} If a=X
We assume that \( b = X \), thus:
\[
X \cdot X = 
\begin{pmatrix} 
0 & 1 \\
1 & 0 
\end{pmatrix}
\begin{pmatrix} 
0 & 1 \\
1 & 0 
\end{pmatrix}
=
\begin{pmatrix} 
1 & 0 \\
0 & 1 
\end{pmatrix}
= Id
\]
Hence this is a group.


\textbf{(b)}[\textbf{Exercise}]
We have

\[
U = e^{-i \Phi_3 Y} e^{-i \Phi_2 X} e^{-i \Phi_1 Y}
\]

\[
\frac{dU}{d\Phi_1} \Big|_{\Phi=0} = -i Y \, e^{-i \Phi_1 Y} e^{-i \Phi_2 X} e^{-i \Phi_3 Y} \Big|_{\Phi=0}
\]

where \(\Phi = 0\) \(\rightarrow\) \(\{\Phi_1, \Phi_2, \Phi_3\} = \{0, 0, 0\}\):

\[
= -i Y
\]

Similarly:

\[
\frac{dU}{d\Phi_2} \Big|_{\Phi=0} = -i X
\]

\[
\frac{dU}{d\Phi_3} \Big|_{\Phi=0} = -i Y
\]

Now we have:

\[
[-i X, -i Y] = -[X, Y]
\]
\[
X = 
\begin{bmatrix}
0 & 1 \\
1 & 0
\end{bmatrix}, \quad
Y = 
\begin{bmatrix}
0 & -i \\
i & 0
\end{bmatrix}
\]

\[
XY = 
\begin{bmatrix}
i & 0 \\
0 & -i
\end{bmatrix}, \quad
YX = 
\begin{bmatrix}
-i & 0 \\
0 & i
\end{bmatrix}
\]

\[
[X, Y] = 2i 
\begin{bmatrix}
1 & 0 \\
0 & -1
\end{bmatrix} = 2i Z, \quad
Z = 
\begin{bmatrix}
1 & 0 \\
0 & -1
\end{bmatrix}
\]

\[
[-i X, -i Y] = 2 (-i Z)
\]

Let:
\[
C_1 = -i X, \quad C_2 = -i Y, \quad C_3 = -i Z
\]

\[
[C_1, C_2] = 2 C_3
\]

Similarly:
\[
[C_1, C_3] = -[X, Z]
\]

\[
XZ = 
\begin{bmatrix}
0 & -1 \\
1 & 0
\end{bmatrix}, \quad
ZX = 
\begin{bmatrix}
0 & 1 \\
-1 & 0
\end{bmatrix}
\]

So:
\[
[C_1, C_3] = -2 (-i Y) = -2 C_2
\]

Similarly:
\[
[C_2, C_3] = -[Y, Z]
\]

\[
YZ = 
\begin{bmatrix}
0 & i \\
i & 0
\end{bmatrix}, \quad
ZY = 
\begin{bmatrix}
0 & -i \\
-i & 0
\end{bmatrix}
\]

So:
\[
[C_2, C_3] = 2(-i X) = 2 C_1
\]

Thus, the commutation relations are:
\[
[C_i , C_j] = 2 \epsilon_{ijk} C_k
\]

Thus, the generators of \(U\) satisfy the same commutation relations as generators of \(SU(2)\).  
Hence, group \(U\) is a representation of the \(SU(2)\) group.


\textbf{(3)}[\textbf{Challenge}]

For any group \(g\), we have:
\[
R(g) \cong \bigoplus_{\lambda} I_{m_\lambda} \otimes R_\lambda(g)
\]

where \(R_\lambda(g)\) are the irreducible representations of group \(g\) and \(m_\lambda\) is their multiplicity.

An element \(A\) of the commutant is of the form:
\[
A \cong  \bigoplus_\lambda A_{m_\lambda} \otimes I_{d_\lambda}
\]

Here \(d_\lambda\) is the dimension of the irreps,  
and \(A_{m_\lambda}\) is an arbitrary \(m_\lambda \times m_\lambda\) matrix.


For an arbitrary \(m_\lambda \times m_\lambda\) matrix, we have \(m_\lambda^2\) basis vectors.


For a matrix given as \(A_{m_\lambda} \otimes I_{d_\lambda}\), we only require \(m_\lambda^2\) basis elements as its basis will be:
\[
\left\{ \{\text{basis element of } A_{m_\lambda}\} \otimes I_{d_\lambda} \right\}
\]

e.g., \(m_\lambda = 2\), \(d_\lambda = 2\), the basis elements are:
\[
\left\{
\begin{bmatrix}
1 & 0 \\
0 & 0
\end{bmatrix} \otimes 
\begin{bmatrix}
1 & 0 \\
0 & 1
\end{bmatrix}, \quad
\begin{bmatrix}
0 & 1 \\
0 & 0
\end{bmatrix} \otimes 
\begin{bmatrix}
1 & 0 \\
0 & 1
\end{bmatrix}, \quad
\begin{bmatrix}
0 & 0 \\
1 & 0
\end{bmatrix} \otimes 
\begin{bmatrix}
1 & 0 \\
0 & 1
\end{bmatrix}, \quad
\begin{bmatrix}
0 & 0 \\
0 & 1
\end{bmatrix} \otimes 
\begin{bmatrix}
1 & 0 \\
0 & 1
\end{bmatrix}
\right\}
\]

Thus, for the element of the commutant \(A\) given as:
\[
A = \bigoplus_\lambda A_{m_\lambda} \otimes I_{d_\lambda}
\]

The number of basis elements or the dimension \(D\) is:
\[
D = \sum_\lambda m_\lambda^2.
\]


\section*{High Dimensional Quantum Computing with Structured Light}

\textbf{(1)} [\textbf{Exercise}] First order Hermite-Gauss modes are given as:

\[
HG_{01}(x, y) = \frac{1}{\omega(z)\sqrt{\pi}} H_0 \left( \frac{\sqrt{2} \, x}{\omega(z)} \right) H_1 \left( \frac{\sqrt{2} \, y}{\omega(z)} \right) 
e^{-\frac{x^2 + y^2}{\omega^2(z)}} e^{i \left( k \frac{(x^2 + y^2)}{2R(z)} \right)} e^{-i \phi_N(z)}
\]

\[
= \frac{\sqrt{2}}{\sqrt{\pi}} \left( \frac{y}{\omega^2(z)} \right) e^{-\frac{x^2 + y^2}{\omega^2(z)}} 
e^{i \left( k \frac{(x^2 + y^2)}{2R(z)} \right)} e^{-i \phi_N(z)}
\]

\[
HG_{10}(x, y) = \frac{1}{\omega(z)\sqrt{\pi}} H_1 \left( \frac{\sqrt{2} \, x}{\omega(z)} \right) H_0 \left( \frac{\sqrt{2} \, y}{\omega(z)} \right) 
e^{-\frac{x^2 + y^2}{\omega^2(z)}} e^{i \left( k \frac{(x^2 + y^2)}{2R(z)} \right)} e^{-i \phi_N(z)}
\]

\[
= \frac{\sqrt{2}}{\sqrt{\pi}} \left( \frac{x}{\omega^2(z)} \right) e^{-\frac{x^2 + y^2}{\omega^2(z)}} 
e^{i \left( k \frac{(x^2 + y^2)}{2R(z)} \right)} e^{-i\phi_N(z)}
\]

where $H_0$, $H_1$ are Hermite Polynomials.

Under a counterclockwise rotation, we have:

\[
x' = \cos \theta \, x + \sin \theta \, y
\]
\[
y' = -\sin \theta \, x + \cos \theta \, y
\]

\[
HG_\theta(x,y) = HG_{10}(x', y') = \frac{\sqrt{2}}{\sqrt{\pi}} \left( \frac{x'}{\omega^2(z)} \right) 
e^{-\frac{x'^2 + y'^2}{\omega^2(z)}} e^{i \left( k \frac{(x'^2 + y'^2)}{2R(z)} \right)} e^{-i \phi_N(z)}
\]

\[
x'^2 + y'^2 = \left( x \cos \theta + y \sin \theta \right)^2 + \left( -x \sin \theta + y \cos \theta \right)^2
\]

\[
= x^2 \cos^2 \theta + y^2 \sin^2 \theta + 2 \cos \theta \sin \theta \, xy 
+ x^2 \sin^2 \theta + y^2 \cos^2 \theta - 2 \cos \theta \sin \theta \, xy
\]

\[
= x^2 + y^2
\]

Thus, 
\[
x'^2 + y'^2 = x^2 + y^2
\]

\[
HG_{\theta}(x,y) = \frac{\sqrt{2}}{\sqrt{\pi} \, \omega^2(z)} \left( \cos \theta \, x + \sin \theta \, y \right) 
e^{-\frac{x^2 + y^2}{\omega^2(z)}} e^{i \left( k \frac{x^2 + y^2}{2 R(z)} \right)} e^{-i \phi_N(z)}
\]

\[
= \cos \theta \, HG_{10}(x, y) + \sin \theta \, HG_{01}(x, y)
\]
\textbf{(a)}
\begin{align*}
\int HG_{\theta}^*(x, y) \, HG_{\theta}(x, y) \, dx \, dy &= 
\int \Big( \cos^2 \theta \, HG_{10}(x, y)^2 + \sin^2 \theta \, HG_{01}(x, y)^2 \\
&+ 2 \sin \theta \cos \theta \, HG_{10}(x, y) \, HG_{01}(x, y) \Big) \, dx \, dy
\end{align*}

\begin{align*}
\int HG_{01}(x, y) \,& HG_{10}(x, y) \, dx \, dy \\
&= \frac{1}{\pi \, \omega^2(z)} 
\left[ \int \left( H_0 \left( \frac{\sqrt{2} \, x}{\omega(z)} \right) \right)^2 
e^{- \frac{x^2}{\omega^2(z)}} \, dx \int \left( H_1 \left( \frac{\sqrt{2} \, y}{\omega(z)} \right) \right)^2 
e^{- \frac{y^2}{\omega^2(z)}} \, dy \right]
\end{align*}

\[
\frac{\sqrt{2} \, x}{\omega(z)} \rightarrow x \quad \text{and} \quad 
\frac{\sqrt{2} \, y}{\omega(z)} \rightarrow y
\]

\[
dx \rightarrow \frac{\omega(z)}{\sqrt{2}} \, dx \quad \text{and} \quad 
dy \rightarrow \frac{\omega(z)}{\sqrt{2}} \, dy
\]

\[
\int \left( HG_{10}(x, y) \right)^2 \, dx \, dy = \frac{1}{2 \pi} 
\int H_0(x) \, H_0(x) \, e^{-x^2} \, dx \int H_1(y) \, H_1(y) \, e^{-y^2} \, dy
\]

Hermite polynomials have the property:
\[
\int H_n(x) \, H_m(x) \, e^{-x^2} \, dx = \sqrt{\pi} \, 2^n \, n! \, \delta_{n m}
\]

Using this, we get:
\[
\int \left( H_0(x) \right)^2 e^{-x^2} \, dx = \sqrt{\pi}
\]

\[
\int \left( H_1(x) \right)^2 e^{-x^2} \, dx = \sqrt{\pi} \, 2
\]
Thus
\[
\int \left( HG_{10}(x, y) \right)^2 \, dx \, dy = 1
\]

Similarly:
\[
\int \left( HG_{01}(x, y) \right)^2 \, dx \, dy = 1
\]

\[
\int HG_{01}(x, y) \, HG_{10}(x, y) \, dx \, dy = \frac{1}{\pi \, \omega^2(z)} 
\left[ \int H_0 \left( \frac{\sqrt{2} \, x}{\omega(z)} \right) 
H_1 \left( \frac{\sqrt{2} \, x}{\omega(z)} \right) 
e^{- \frac{2 x^2}{\omega^2(z)}} \, dx \right]
\]

\[
\times \left[ \int H_1 \left( \frac{\sqrt{2} \, y}{\omega(z)} \right) 
H_0 \left( \frac{\sqrt{2} \, y}{\omega(z)} \right) 
e^{- \frac{2 y^2}{\omega^2(z)}} \, dy \right]
\]

\[
= \frac{1}{2 \pi} \int H_0(x) \, H_1(x) \, e^{-x^2} \, dx 
\int H_1(y) \, H_0(y) \, e^{-y^2} \, dy
\]

\[
= 0
\]

Thus:
\[
\int HG_{\theta}^*(x, y) \, HG_{\theta}(x, y) \, dx \, dy = \cos^2 \theta + \sin^2 \theta = 1
\]

\textbf{(b)}
\[
\int HG_{\theta}^*(x, y) \, HG_{\theta + \pi/2}(x, y) \, dx \, dy
\]

\[
= \int \left( \cos \theta \, HG_{10}(x, y) + \sin \theta \, HG_{01}(x, y) \right)
\left( -\sin \theta \, HG_{10}(x, y) + \cos \theta \, HG_{01}(x, y) \right) \, dx \, dy
\]

\begin{align*}
    = \int \Big[ -\cos \theta \sin \theta \, HG_{10}(x, y)^2 +& \cos^2 \theta \sin^2 \theta \, HG_{01}(x, y)^2 \\
    &
+ (\cos^2 \theta - \sin^2 \theta) \, HG_{01}(x, y) \, HG_{10}(x, y) \Big] \, dx \, dy
\end{align*}


Using previous results:
\[
\int HG_{\theta}^*(x, y) \, HG_{\theta + \pi/2}(x, y) \, dx \, dy= -\cos \theta \sin \theta + \sin \theta \cos \theta = 0
\]

\textbf{(c)}
\[
LG_{0,\pm1}(r,\varphi) = \frac{1}{\sqrt{2\pi} \, \omega(z)} \left( \frac{\sqrt{2} \, (x^2 + y^2)}{\omega(z)} \right) 
L_0 \left( \frac{2 (x^2 + y^2)}{\omega^2(z)} \right) e^{-\frac{x^2 + y^2}{\omega^2(z)}} e^{i \left( k \frac{x^2 + y^2}{2R(z)} \right)} e^{-i \phi_N(z)} e^{\pm i l \varphi}
\]



\[
= \frac{x^2 + y^2}{\sqrt{\pi} \, \omega(z)} e^{-\frac{x^2 + y^2}{\omega^2(z)}}
e^{\left( i k \frac{x^2 + y^2}{2R(z)} \right)} e^{-i \phi_0(z)} e^{i \ell \varphi}
\]

with \(x = r \cos \varphi\) and \(y = r \sin \varphi\), we get:

\[
= \frac{1}{\sqrt{\pi} \, \omega(z)^2} \left( r \cos \varphi + i r \sin \varphi \right)
e^{-\frac{x^2 + y^2}{\omega^2(z)}} e^{\left( i k \frac{x^2 + y^2}{2R(z)} \right)} e^{-i \phi_N(z)}
\]

\[
= \frac{1}{\sqrt{2}} \left[ \frac{\sqrt{2}}{\sqrt{\pi} \, \omega(z)^2} 
x \, e^{-\frac{x^2 + y^2}{\omega^2(z)}} e^{\left( i k \frac{x^2 + y^2}{2R(z)} \right)} e^{-i \phi_N(z)} \right.
\]

\[
\left. \quad \quad \quad \quad \quad \quad +\frac{\sqrt{2}}{\sqrt{\pi} \, \omega(z)^2} 
i y \, e^{-\frac{x^2 + y^2}{\omega^2(z)}} e^{\left( i k \frac{x^2 + y^2}{2R(z)} \right)} e^{-i \phi_N(z)} \right]
\]

\[
= \frac{HG_{10} \, + \, i \, HG_{01}}{\sqrt{2}}
\]

\textbf{(2)} [\textbf{Exercise}]:
\[
\sum_{n,m=-\infty}^{\infty} u_{n,m}(x, y) \, u_{n,m}^*(x', y')
= \sum_{n,m} \langle x, y | u_{n,m} \rangle \langle u_{n,m} | x', y' \rangle
= \langle x, y | \left( \sum_{n,m} | u_{n,m} \rangle \langle u_{n,m} | \right) | x', y' \rangle
\]

Since \(u_{n,m}\) is a basis set, it is complete.  
Hence:
\[
\sum_{n,m} | u_{n,m} \rangle \langle u_{n,m} | = I
\]

Thus, we get:
\[
\sum_{n,m=-\infty}^{\infty} u_{n,m}(x, y) \, u_{n,m}^*(x', y') = \langle x, y | x', y' \rangle
\]

\(\{x, y\}\) also form a complete orthogonal basis in \(\mathbb{R}^2\).  
Thus:
\[
\langle x, y | x', y' \rangle = \delta(x - x') \, \delta(y - y')
\]

Hence:
\[
\sum_{n,m} u_{n,m}(x, y) \, u_{n,m}^*(x', y') = \delta(x - x') \, \delta(y - y')
\]

\textbf{(3)}[\textbf{Challenge}]
\[
\Psi_{\theta}(x, y) = HG_{\theta}(x, y) \, \hat{e}_{\theta} + HG_{\frac{\pi}{2}+\theta}(x, y) \, \hat{e}_{\frac{\pi}{2}+\theta}
\]

\[
= HG_{\theta}(x, y) \left( \cos \theta \, \hat{e}_H + \sin \theta \, \hat{e}_V \right)
+ HG_{\frac{\pi}{2}+\theta}(x, y) \left( -\sin \theta \, \hat{e}_H + \cos \theta \, \hat{e}_V \right)
\]

\[
= \left( \cos \theta \, HG_{\theta}(x, y) - \sin \theta \, HG_{\frac{\pi}{2}+\theta}(x, y) \right) \hat{e}_H 
+ \left( \sin \theta \, HG_{\theta}(x, y) + \cos \theta \, HG_{\frac{\pi}{2}+\theta}(x, y) \right) \hat{e}_V
\]

\[
= \left[ \cos \theta \left( \cos \theta \, HG_{10}(x,y) + \sin \theta \, HG_{01}(x,y) \right) 
- \sin \theta \left( -\sin \theta \, HG_{10}(x,y) + \cos \theta \, HG_{01}(x,y) \right) \right] \hat{e}_H
\]

\[
+ \left[ \sin \theta \left( \cos \theta \, HG_{10}(x,y) + \sin \theta \, HG_{01}(x,y) \right) 
+ \cos \theta \left( -\sin \theta \, HG_{10}(x,y) + \cos \theta \, HG_{01}(x,y) \right) \right] \hat{e}_V
\]

\[
= HG_{10}(x,y) \, \hat{e}_H + HG_{01}(x,y) \, \hat{e}_V
\]

Hence, it remains invariant under rotation.\\

Similarly:

\[
\Psi_{\theta}(x, y) = HG_{\theta}(x, y) \, \hat{e}_{\frac{\pi}{2}+\theta} - HG_{\frac{\pi}{2}+\theta}(x, y) \, \hat{e}_{\theta}
\]

\[
= \left( \cos \theta \, HG_{10}(x,y) + \sin \theta \, HG_{01}(x,y) \right) \left( -\sin \theta \, \hat{e}_H + \cos \theta \, \hat{e}_V \right)
\]

\[
- \left( -\sin \theta \, HG_{10}(x,y) + \cos \theta \, HG_{01}(x,y) \right) \left( \cos \theta \, \hat{e}_H + \sin \theta \, \hat{e}_V \right)
\]

\[
= \hat{e}_H \left[ -\sin \theta \cos \theta \, HG_{10}(x,y) - \sin^2 \theta \, HG_{01}(x,y) 
+ \sin \theta \cos \theta \, HG_{10}(x,y) - \cos^2 \theta \, HG_{01}(x,y) \right]
\]

\[
+ \hat{e}_V \left[ \cos^2 \theta \, HG_{10}(x,y) + \cos \theta \sin \theta \, HG_{01}(x,y) 
+ \sin^2 \theta \, HG_{10}(x,y) - \sin \theta \cos \theta \, HG_{01}(x,y) \right]
\]

\[
= -\left( HG_{01}(x,y) \, \hat{e}_H - HG_{10}(x,y) \, \hat{e}_V \right)
\]

Hence, this is also rotationally invariant up to a global phase.

\textbf{(b)} We have from (a):
\[
\Psi(x, y) = HG_{\theta}(x,y) \, \hat{e}_{\theta} + HG_{\frac{\pi}{2}+\theta}(x,y) \, \hat{e}_{\frac{\pi}{2}+\theta}
\]
\[
= HG_{10}(x,y) \, \hat{e}_H + HG_{01}(x,y) \, \hat{e}_V
\]

\[
I_H = \int \left| \hat{e}_H^* \cdot \Psi(x, y) \right|^2 \, dx \, dy 
= \int \left| HG_{10} \right|^2 \, dx \, dy = \int HG_{10}^2 \, dx \, dy = 1
\]

\[
I_V = \int \left| \hat{e}_V^* \cdot \Psi(x, y) \right|^2 \, dx \, dy 
= \int \left| HG_{01} \right|^2 \, dx \, dy = \int HG_{01}^2 \, dx \, dy = 1
\]

\[
\hat{e}_H = \frac{\hat{e}_0 + \hat{e}_A}{\sqrt{2}}, \quad 
\hat{e}_V = \frac{\hat{e}_0 - \hat{e}_A}{\sqrt{2}}
\]

\[
\Psi_{\theta}(x, y) = \frac{(HG_{10} + HG_{01})}{\sqrt{2}} \, \hat{e}_0 
+ \frac{(HG_{10} - HG_{01})}{\sqrt{2}} \, \hat{e}_A
\]

\[
I_D = \int \left| \frac{HG_{10} + HG_{01}}{\sqrt{2}} \right|^2 \, dx \, dy 
= \frac{1}{2} \int \left( HG_{10}^2 + HG_{01}^2 + 2 \, HG_{10} \, HG_{01} \right) \, dx \, dy
\]

\[
= \frac{1}{2} \left[ 1 + 1 + 0 \right] = 1
\]

\[
I_A = \int \left| \frac{HG_{10} - HG_{01}}{\sqrt{2}} \right|^2 \, dx \, dy 
= \frac{1}{2} \int \left( HG_{10}^2 + HG_{01}^2 - 2 \, HG_{10} \, HG_{01} \right) \, dx \, dy
\]

\[
= \frac{1}{2} \left[ 1 + 1 - 0 \right] = 1
\]

\[
\hat{e}_H = \frac{\hat{e}_L + i \hat{e}_R}{\sqrt{2}}, \quad 
\hat{e}_V = \frac{\hat{e}_L - i \hat{e}_R}{\sqrt{2}}
\]

\[
\Psi_{\theta}(x, y) = \frac{(HG_{10} + i \, HG_{01})}{\sqrt{2}} \, \hat{e}_L 
+ \frac{(HG_{10} - i \, HG_{01})}{\sqrt{2}} \, \hat{e}_R
\]

\[
I_L = \int \left| \frac{HG_{10} + i \, HG_{01}}{\sqrt{2}} \right|^2 \, dx \, dy 
= \frac{1}{2} \int \left( HG_{10}^2 + HG_{01}^2 \right) \, dx \, dy = 1
\]

\[
I_R = \int \left| \frac{HG_{10} - i \, HG_{01}}{\sqrt{2}} \right|^2 \, dx \, dy 
= \frac{1}{2} \int \left( HG_{10}^2 + HG_{01}^2 \right) \, dx \, dy = 1
\]
The polarisation Stokes parameters are thus
\[
S_1 = \frac{I_H - I_V}{I_{Tot}} = 0, \quad 
S_2 = \frac{I_D - I_A}{I_{Tot}} = 0, \quad 
S_3 = \frac{I_L - I_R}{I_{Tot}} = 0
\]
all zero.\\


The large detectors are required so the whole beam of light can be covered.  
Thus, integration over the whole space can be done to use Hermite-Gauss polynomial’s orthogonality properties.\\



Similarly for:
\[
\Psi_{\theta}(x, y) = HG_{\theta}(x, y) \, \hat{e}_{\frac{\pi}{2}+\theta} - HG_{\frac{\pi}{2}+\theta}(x, y) \, \hat{e}_{\theta} 
\]
\[
= -HG_{01} \, \hat{e}_H + HG_{10} \, \hat{e}_V
\]

Similarly to previous cases:
\[
I_H = \int HG_{10}^2 \, dx \, dy = 1, \quad 
I_V = \int HG_{01}^2 \, dx \, dy = 1
\]

\[
I_D = \int \left| \frac{HG_{10} + HG_{01}}{\sqrt{2}} \right|^2 \, dx \, dy = 1, \quad 
I_A = \int \left| \frac{HG_{10} - HG_{01}}{\sqrt{2}} \right|^2 \, dx \, dy = 1
\]

\[
I_L = \int \left| \frac{HG_{10} - i \, HG_{01}}{\sqrt{2}} \right|^2 \, dx \, dy = 1, \quad 
I_R = \int \left| \frac{HG_{10} + i \, HG_{01}}{\sqrt{2}} \right|^2 \, dx \, dy = 1
\]
Thus again we have the polarisation Stokes parameters
\[
S_1 = \frac{I_H - I_V}{I_{Tot}} = 0, \quad 
S_2 = \frac{I_D - I_A}{I_{Tot}} = 0, \quad 
S_3 = \frac{I_L - I_R}{I_{Tot}} = 0
\]
to be zero
\end{document}